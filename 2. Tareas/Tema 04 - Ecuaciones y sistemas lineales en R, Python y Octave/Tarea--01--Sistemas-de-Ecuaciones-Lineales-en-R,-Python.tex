% Options for packages loaded elsewhere
\PassOptionsToPackage{unicode}{hyperref}
\PassOptionsToPackage{hyphens}{url}
%
\documentclass[
]{article}
\usepackage{lmodern}
\usepackage{amssymb,amsmath}
\usepackage{ifxetex,ifluatex}
\ifnum 0\ifxetex 1\fi\ifluatex 1\fi=0 % if pdftex
  \usepackage[T1]{fontenc}
  \usepackage[utf8]{inputenc}
  \usepackage{textcomp} % provide euro and other symbols
\else % if luatex or xetex
  \usepackage{unicode-math}
  \defaultfontfeatures{Scale=MatchLowercase}
  \defaultfontfeatures[\rmfamily]{Ligatures=TeX,Scale=1}
\fi
% Use upquote if available, for straight quotes in verbatim environments
\IfFileExists{upquote.sty}{\usepackage{upquote}}{}
\IfFileExists{microtype.sty}{% use microtype if available
  \usepackage[]{microtype}
  \UseMicrotypeSet[protrusion]{basicmath} % disable protrusion for tt fonts
}{}
\makeatletter
\@ifundefined{KOMAClassName}{% if non-KOMA class
  \IfFileExists{parskip.sty}{%
    \usepackage{parskip}
  }{% else
    \setlength{\parindent}{0pt}
    \setlength{\parskip}{6pt plus 2pt minus 1pt}}
}{% if KOMA class
  \KOMAoptions{parskip=half}}
\makeatother
\usepackage{xcolor}
\IfFileExists{xurl.sty}{\usepackage{xurl}}{} % add URL line breaks if available
\IfFileExists{bookmark.sty}{\usepackage{bookmark}}{\usepackage{hyperref}}
\hypersetup{
  pdftitle={Tareas de Sistemas de Ecuaciones Lineales en R y Python},
  pdfauthor={Ramon Ceballos},
  hidelinks,
  pdfcreator={LaTeX via pandoc}}
\urlstyle{same} % disable monospaced font for URLs
\usepackage[margin=1in]{geometry}
\usepackage{color}
\usepackage{fancyvrb}
\newcommand{\VerbBar}{|}
\newcommand{\VERB}{\Verb[commandchars=\\\{\}]}
\DefineVerbatimEnvironment{Highlighting}{Verbatim}{commandchars=\\\{\}}
% Add ',fontsize=\small' for more characters per line
\usepackage{framed}
\definecolor{shadecolor}{RGB}{248,248,248}
\newenvironment{Shaded}{\begin{snugshade}}{\end{snugshade}}
\newcommand{\AlertTok}[1]{\textcolor[rgb]{0.94,0.16,0.16}{#1}}
\newcommand{\AnnotationTok}[1]{\textcolor[rgb]{0.56,0.35,0.01}{\textbf{\textit{#1}}}}
\newcommand{\AttributeTok}[1]{\textcolor[rgb]{0.77,0.63,0.00}{#1}}
\newcommand{\BaseNTok}[1]{\textcolor[rgb]{0.00,0.00,0.81}{#1}}
\newcommand{\BuiltInTok}[1]{#1}
\newcommand{\CharTok}[1]{\textcolor[rgb]{0.31,0.60,0.02}{#1}}
\newcommand{\CommentTok}[1]{\textcolor[rgb]{0.56,0.35,0.01}{\textit{#1}}}
\newcommand{\CommentVarTok}[1]{\textcolor[rgb]{0.56,0.35,0.01}{\textbf{\textit{#1}}}}
\newcommand{\ConstantTok}[1]{\textcolor[rgb]{0.00,0.00,0.00}{#1}}
\newcommand{\ControlFlowTok}[1]{\textcolor[rgb]{0.13,0.29,0.53}{\textbf{#1}}}
\newcommand{\DataTypeTok}[1]{\textcolor[rgb]{0.13,0.29,0.53}{#1}}
\newcommand{\DecValTok}[1]{\textcolor[rgb]{0.00,0.00,0.81}{#1}}
\newcommand{\DocumentationTok}[1]{\textcolor[rgb]{0.56,0.35,0.01}{\textbf{\textit{#1}}}}
\newcommand{\ErrorTok}[1]{\textcolor[rgb]{0.64,0.00,0.00}{\textbf{#1}}}
\newcommand{\ExtensionTok}[1]{#1}
\newcommand{\FloatTok}[1]{\textcolor[rgb]{0.00,0.00,0.81}{#1}}
\newcommand{\FunctionTok}[1]{\textcolor[rgb]{0.00,0.00,0.00}{#1}}
\newcommand{\ImportTok}[1]{#1}
\newcommand{\InformationTok}[1]{\textcolor[rgb]{0.56,0.35,0.01}{\textbf{\textit{#1}}}}
\newcommand{\KeywordTok}[1]{\textcolor[rgb]{0.13,0.29,0.53}{\textbf{#1}}}
\newcommand{\NormalTok}[1]{#1}
\newcommand{\OperatorTok}[1]{\textcolor[rgb]{0.81,0.36,0.00}{\textbf{#1}}}
\newcommand{\OtherTok}[1]{\textcolor[rgb]{0.56,0.35,0.01}{#1}}
\newcommand{\PreprocessorTok}[1]{\textcolor[rgb]{0.56,0.35,0.01}{\textit{#1}}}
\newcommand{\RegionMarkerTok}[1]{#1}
\newcommand{\SpecialCharTok}[1]{\textcolor[rgb]{0.00,0.00,0.00}{#1}}
\newcommand{\SpecialStringTok}[1]{\textcolor[rgb]{0.31,0.60,0.02}{#1}}
\newcommand{\StringTok}[1]{\textcolor[rgb]{0.31,0.60,0.02}{#1}}
\newcommand{\VariableTok}[1]{\textcolor[rgb]{0.00,0.00,0.00}{#1}}
\newcommand{\VerbatimStringTok}[1]{\textcolor[rgb]{0.31,0.60,0.02}{#1}}
\newcommand{\WarningTok}[1]{\textcolor[rgb]{0.56,0.35,0.01}{\textbf{\textit{#1}}}}
\usepackage{graphicx,grffile}
\makeatletter
\def\maxwidth{\ifdim\Gin@nat@width>\linewidth\linewidth\else\Gin@nat@width\fi}
\def\maxheight{\ifdim\Gin@nat@height>\textheight\textheight\else\Gin@nat@height\fi}
\makeatother
% Scale images if necessary, so that they will not overflow the page
% margins by default, and it is still possible to overwrite the defaults
% using explicit options in \includegraphics[width, height, ...]{}
\setkeys{Gin}{width=\maxwidth,height=\maxheight,keepaspectratio}
% Set default figure placement to htbp
\makeatletter
\def\fps@figure{htbp}
\makeatother
\setlength{\emergencystretch}{3em} % prevent overfull lines
\providecommand{\tightlist}{%
  \setlength{\itemsep}{0pt}\setlength{\parskip}{0pt}}
\setcounter{secnumdepth}{-\maxdimen} % remove section numbering

\title{Tareas de Sistemas de Ecuaciones Lineales en R y Python}
\author{Ramon Ceballos}
\date{24/2/2021}

\begin{document}
\maketitle

\hypertarget{ejercicio-1}{%
\section{Ejercicio 1}\label{ejercicio-1}}

Resolver el siguiente sistema del pdf.

Primero, comprobar el tipo de sistema (compatible determinado,
compatible indeterminado o incompatible) con R, Python y Octave.

Después, en caso de haber solución, calcularla con R, Python y Octave.
Finalmente, indicar la solución final junto con el procedimiento llevado
a cabo.

\begin{itemize}
\tightlist
\item
  \textbf{R}
\end{itemize}

\begin{Shaded}
\begin{Highlighting}[]
\KeywordTok{library}\NormalTok{(matlib)}
\end{Highlighting}
\end{Shaded}

Definimos las matrices de coeficientes, de términos independientes y
ampliada.

\begin{Shaded}
\begin{Highlighting}[]
\NormalTok{A =}\StringTok{ }\KeywordTok{rbind}\NormalTok{(}\KeywordTok{c}\NormalTok{(}\DecValTok{10}\NormalTok{,}\DecValTok{2}\NormalTok{,}\OperatorTok{-}\DecValTok{1}\NormalTok{,}\DecValTok{1}\NormalTok{,}\DecValTok{0}\NormalTok{,}\DecValTok{10}\NormalTok{), }\KeywordTok{c}\NormalTok{(}\OperatorTok{-}\DecValTok{1}\NormalTok{,}\OperatorTok{-}\DecValTok{3}\NormalTok{,}\DecValTok{0}\NormalTok{,}\DecValTok{0}\NormalTok{,}\OperatorTok{-}\DecValTok{1}\NormalTok{,}\DecValTok{5}\NormalTok{), }\KeywordTok{c}\NormalTok{(}\DecValTok{0}\NormalTok{,}\OperatorTok{-}\DecValTok{1}\NormalTok{,}\DecValTok{3}\NormalTok{,}\OperatorTok{-}\DecValTok{1}\NormalTok{,}\DecValTok{0}\NormalTok{,}\DecValTok{0}\NormalTok{), }\KeywordTok{c}\NormalTok{(}\DecValTok{17}\NormalTok{,}\DecValTok{1}\NormalTok{,}\DecValTok{0}\NormalTok{,}\DecValTok{3}\NormalTok{,}\DecValTok{5}\NormalTok{,}\OperatorTok{-}\DecValTok{15}\NormalTok{), }\KeywordTok{c}\NormalTok{(}\DecValTok{0}\NormalTok{,}\OperatorTok{-}\DecValTok{10}\NormalTok{,}\DecValTok{0}\NormalTok{,}\OperatorTok{-}\DecValTok{5}\NormalTok{,}\DecValTok{3}\NormalTok{,}\DecValTok{0}\NormalTok{), }\KeywordTok{c}\NormalTok{(}\OperatorTok{-}\DecValTok{3}\NormalTok{,}\DecValTok{1}\NormalTok{,}\DecValTok{1}\NormalTok{,}\DecValTok{1}\NormalTok{,}\OperatorTok{-}\DecValTok{2}\NormalTok{,}\DecValTok{2}\NormalTok{))}

\NormalTok{b =}\StringTok{ }\KeywordTok{c}\NormalTok{(}\DecValTok{0}\NormalTok{,}\DecValTok{5}\NormalTok{,}\DecValTok{5}\NormalTok{,}\DecValTok{4}\NormalTok{,}\OperatorTok{-}\DecValTok{21}\NormalTok{,}\DecValTok{11}\NormalTok{)}

\NormalTok{AB =}\StringTok{ }\KeywordTok{cbind}\NormalTok{(A,b)}
\end{Highlighting}
\end{Shaded}

Vemos si el rango de la ampliada es igual que el de la matriz de
coeficientes; y si es igual al nº de incognitas.

\begin{Shaded}
\begin{Highlighting}[]
\KeywordTok{R}\NormalTok{(A) }\OperatorTok{==}\StringTok{ }\KeywordTok{R}\NormalTok{(AB)}
\end{Highlighting}
\end{Shaded}

\begin{verbatim}
## [1] TRUE
\end{verbatim}

\begin{Shaded}
\begin{Highlighting}[]
\KeywordTok{R}\NormalTok{(A) }\OperatorTok{==}\StringTok{ }\DecValTok{6}
\end{Highlighting}
\end{Shaded}

\begin{verbatim}
## [1] TRUE
\end{verbatim}

Vemos que ambas hipótesis se cumple por lo que estamos ante un sistema
compatible determinado.

Ahora procedemos a encontrar cual es la solución al sistema.

\begin{Shaded}
\begin{Highlighting}[]
\KeywordTok{Solve}\NormalTok{(A,b, }\DataTypeTok{fractions=}\OtherTok{TRUE}\NormalTok{)}
\end{Highlighting}
\end{Shaded}

\begin{verbatim}
## x1            =   0 
##   x2          =  -1 
##     x3        =   3 
##       x4      =   5 
##         x5    =  -2 
##           x6  =   0
\end{verbatim}

Por tanto, la solución del sistema es
\texttt{S\ =\ c(0,-1,\ 3,\ 5,\ -2,\ 0)}.

\begin{itemize}
\tightlist
\item
  \textbf{Python}
\end{itemize}

\begin{Shaded}
\begin{Highlighting}[]
\ImportTok{import}\NormalTok{ numpy }\ImportTok{as}\NormalTok{ np}
\end{Highlighting}
\end{Shaded}

Definimos las matrices de coeficientes, de términos independientes y
ampliada.

\begin{Shaded}
\begin{Highlighting}[]
\NormalTok{A }\OperatorTok{=}\NormalTok{ np.array([[}\DecValTok{10}\NormalTok{,}\DecValTok{2}\NormalTok{,}\OperatorTok{-}\DecValTok{1}\NormalTok{,}\DecValTok{1}\NormalTok{,}\DecValTok{0}\NormalTok{,}\DecValTok{10}\NormalTok{],[}\OperatorTok{-}\DecValTok{1}\NormalTok{,}\OperatorTok{-}\DecValTok{3}\NormalTok{,}\DecValTok{0}\NormalTok{,}\DecValTok{0}\NormalTok{,}\OperatorTok{-}\DecValTok{1}\NormalTok{,}\DecValTok{5}\NormalTok{],[}\DecValTok{0}\NormalTok{,}\OperatorTok{-}\DecValTok{1}\NormalTok{,}\DecValTok{3}\NormalTok{,}\OperatorTok{-}\DecValTok{1}\NormalTok{,}\DecValTok{0}\NormalTok{,}\DecValTok{0}\NormalTok{],[}\DecValTok{17}\NormalTok{,}\DecValTok{1}\NormalTok{,}\DecValTok{0}\NormalTok{,}\DecValTok{3}\NormalTok{,}\DecValTok{5}\NormalTok{,}\OperatorTok{-}\DecValTok{15}\NormalTok{],[}\DecValTok{0}\NormalTok{,}\OperatorTok{-}\DecValTok{10}\NormalTok{,}\DecValTok{0}\NormalTok{,}\OperatorTok{-}\DecValTok{5}\NormalTok{,}\DecValTok{3}\NormalTok{,}\DecValTok{0}\NormalTok{],[}\OperatorTok{-}\DecValTok{3}\NormalTok{,}\DecValTok{1}\NormalTok{,}\DecValTok{1}\NormalTok{,}\DecValTok{1}\NormalTok{,}\OperatorTok{-}\DecValTok{2}\NormalTok{,}\DecValTok{2}\NormalTok{]])}

\NormalTok{b }\OperatorTok{=}\NormalTok{ np.array([}\DecValTok{0}\NormalTok{,}\DecValTok{5}\NormalTok{,}\DecValTok{5}\NormalTok{,}\DecValTok{4}\NormalTok{,}\OperatorTok{-}\DecValTok{21}\NormalTok{,}\DecValTok{11}\NormalTok{])}

\NormalTok{AB }\OperatorTok{=}\NormalTok{ np.array([[}\DecValTok{10}\NormalTok{,}\DecValTok{2}\NormalTok{,}\OperatorTok{-}\DecValTok{1}\NormalTok{,}\DecValTok{1}\NormalTok{,}\DecValTok{0}\NormalTok{,}\DecValTok{10}\NormalTok{,}\DecValTok{0}\NormalTok{],[}\OperatorTok{-}\DecValTok{1}\NormalTok{,}\OperatorTok{-}\DecValTok{3}\NormalTok{,}\DecValTok{0}\NormalTok{,}\DecValTok{0}\NormalTok{,}\OperatorTok{-}\DecValTok{1}\NormalTok{,}\DecValTok{5}\NormalTok{,}\DecValTok{5}\NormalTok{],[}\DecValTok{0}\NormalTok{,}\OperatorTok{-}\DecValTok{1}\NormalTok{,}\DecValTok{3}\NormalTok{,}\OperatorTok{-}\DecValTok{1}\NormalTok{,}\DecValTok{0}\NormalTok{,}\DecValTok{0}\NormalTok{,}\DecValTok{5}\NormalTok{],[}\DecValTok{17}\NormalTok{,}\DecValTok{1}\NormalTok{,}\DecValTok{0}\NormalTok{,}\DecValTok{3}\NormalTok{,}\DecValTok{5}\NormalTok{,}\OperatorTok{-}\DecValTok{15}\NormalTok{,}\DecValTok{4}\NormalTok{],[}\DecValTok{0}\NormalTok{,}\OperatorTok{-}\DecValTok{10}\NormalTok{,}\DecValTok{0}\NormalTok{,}\OperatorTok{-}\DecValTok{5}\NormalTok{,}\DecValTok{3}\NormalTok{,}\DecValTok{0}\NormalTok{,}\OperatorTok{-}\DecValTok{21}\NormalTok{],[}\OperatorTok{-}\DecValTok{3}\NormalTok{,}\DecValTok{1}\NormalTok{,}\DecValTok{1}\NormalTok{,}\DecValTok{1}\NormalTok{,}\OperatorTok{-}\DecValTok{2}\NormalTok{,}\DecValTok{2}\NormalTok{,}\DecValTok{11}\NormalTok{]])}
\end{Highlighting}
\end{Shaded}

Vemos si el rango de la ampliada es igual que el de la matriz de
coeficientes; y si es igual al nº de incognitas.

\begin{Shaded}
\begin{Highlighting}[]
\NormalTok{np.linalg.matrix_rank(A) }\OperatorTok{==}\NormalTok{ np.linalg.matrix_rank(AB)}
\end{Highlighting}
\end{Shaded}

\begin{verbatim}
## True
\end{verbatim}

\begin{Shaded}
\begin{Highlighting}[]
\NormalTok{np.linalg.matrix_rank(A) }\OperatorTok{==} \DecValTok{6}
\end{Highlighting}
\end{Shaded}

\begin{verbatim}
## True
\end{verbatim}

Vemos que ambas hipótesis se cumple por lo que estamos ante un sistema
compatible determinado.

Ahora procedemos a encontrar cual es la solución al sistema.

\begin{Shaded}
\begin{Highlighting}[]
\NormalTok{np.}\BuiltInTok{round}\NormalTok{(np.linalg.solve(A,b),}\DecValTok{3}\NormalTok{)}
\end{Highlighting}
\end{Shaded}

\begin{verbatim}
## array([ 0., -1.,  3.,  5., -2., -0.])
\end{verbatim}

Tenemos que el vector solución final obtenido es
\texttt{S\ =\ {[}0,-1,\ 3,\ 5,\ -2,\ 0{]}}.

\hypertarget{ejercicio-2}{%
\section{Ejercicio 2}\label{ejercicio-2}}

Resolver el siguiente sistema del pdf.

Primero, comprobar el tipo de sistema (compatible determinado,
compatible indeterminado o incompatible) con R, Python y Octave.

Después, en caso de haber solución, calcularla con R, Python y Octave.
Finalmente, indicar la solución final junto con el procedimiento llevado
a cabo.

\begin{itemize}
\tightlist
\item
  \textbf{R}
\end{itemize}

Definimos las matrices de coeficientes, de términos independientes y
ampliada.

\begin{Shaded}
\begin{Highlighting}[]
\NormalTok{A =}\StringTok{ }\KeywordTok{rbind}\NormalTok{(}\KeywordTok{c}\NormalTok{(}\OperatorTok{-}\DecValTok{2}\NormalTok{,}\DecValTok{2}\NormalTok{,}\OperatorTok{-}\DecValTok{1}\NormalTok{,}\DecValTok{1}\NormalTok{,}\DecValTok{0}\NormalTok{,}\DecValTok{4}\NormalTok{,}\DecValTok{0}\NormalTok{), }\KeywordTok{c}\NormalTok{(}\OperatorTok{-}\DecValTok{1}\NormalTok{,}\OperatorTok{-}\DecValTok{3}\NormalTok{,}\DecValTok{0}\NormalTok{,}\DecValTok{0}\NormalTok{,}\OperatorTok{-}\DecValTok{1}\NormalTok{,}\DecValTok{5}\NormalTok{,}\OperatorTok{-}\DecValTok{2}\NormalTok{), }\KeywordTok{c}\NormalTok{(}\DecValTok{0}\NormalTok{,}\OperatorTok{-}\DecValTok{1}\NormalTok{,}\DecValTok{3}\NormalTok{,}\OperatorTok{-}\DecValTok{1}\NormalTok{,}\DecValTok{0}\NormalTok{,}\DecValTok{0}\NormalTok{,}\DecValTok{0}\NormalTok{), }\KeywordTok{c}\NormalTok{(}\DecValTok{0}\NormalTok{,}\DecValTok{1}\NormalTok{,}\DecValTok{0}\NormalTok{,}\DecValTok{3}\NormalTok{,}\OperatorTok{-}\DecValTok{2}\NormalTok{,}\DecValTok{1}\NormalTok{,}\DecValTok{4}\NormalTok{), }\KeywordTok{c}\NormalTok{(}\DecValTok{0}\NormalTok{,}\OperatorTok{-}\DecValTok{3}\NormalTok{,}\DecValTok{0}\NormalTok{,}\OperatorTok{-}\DecValTok{5}\NormalTok{,}\DecValTok{3}\NormalTok{,}\DecValTok{0}\NormalTok{,}\OperatorTok{-}\DecValTok{2}\NormalTok{), }\KeywordTok{c}\NormalTok{(}\OperatorTok{-}\DecValTok{3}\NormalTok{,}\DecValTok{1}\NormalTok{,}\DecValTok{1}\NormalTok{,}\DecValTok{1}\NormalTok{,}\OperatorTok{-}\DecValTok{2}\NormalTok{,}\DecValTok{2}\NormalTok{,}\DecValTok{1}\NormalTok{))}

\NormalTok{b =}\StringTok{ }\KeywordTok{c}\NormalTok{(}\DecValTok{5}\NormalTok{,}\DecValTok{3}\NormalTok{,}\DecValTok{5}\NormalTok{,}\DecValTok{0}\NormalTok{,}\DecValTok{5}\NormalTok{,}\DecValTok{0}\NormalTok{)}

\NormalTok{AB =}\StringTok{ }\KeywordTok{cbind}\NormalTok{(A,b)}
\end{Highlighting}
\end{Shaded}

Vemos si el rango de la ampliada es igual que el de la matriz de
coeficientes; y si es igual al nº de incognitas.

\begin{Shaded}
\begin{Highlighting}[]
\KeywordTok{R}\NormalTok{(A) }\OperatorTok{==}\StringTok{ }\KeywordTok{R}\NormalTok{(AB)}
\end{Highlighting}
\end{Shaded}

\begin{verbatim}
## [1] TRUE
\end{verbatim}

\begin{Shaded}
\begin{Highlighting}[]
\KeywordTok{R}\NormalTok{(A) }\OperatorTok{==}\StringTok{ }\DecValTok{7}
\end{Highlighting}
\end{Shaded}

\begin{verbatim}
## [1] FALSE
\end{verbatim}

Vemos que los rangos de las matrices comparadas son iguales, sin embargo
el número de incógnitas es diferente al rango por lo que estamos ante un
sistema compatible indeterminado.

Ahora procedemos a encontrar cual es la solución al sistema.

\begin{Shaded}
\begin{Highlighting}[]
\KeywordTok{Solve}\NormalTok{(A,b, }\DataTypeTok{fractions=}\OtherTok{TRUE}\NormalTok{)}
\end{Highlighting}
\end{Shaded}

\begin{verbatim}
## x1             + 621/7*x7  =     467/7 
##   x2           + 509/7*x7  =     383/7 
##     x3        - 233/14*x7  =   -149/14 
##       x4     - 1717/14*x7  =  -1283/14 
##         x5   - 1853/14*x7  =  -1349/14 
##           x6    + 69/2*x7  =      55/2
\end{verbatim}

Por tanto, la solución del sistema es S =
(\(\frac{467}{7} - \frac{621x_7}{7}\),\(\frac{383}{7} - \frac{509x_7}{7}\),
\(\frac{-149}{14} + \frac{233x_7}{14}\),
\(\frac{-1283}{14} + \frac{1717x_7}{14}\),
\(\frac{-1349}{14} + \frac{1853x_7}{14}\),
\(\frac{55}{2} - \frac{69x_7}{2}\), \(x_7\)), siendo \(x_7\) la variable
indeterminada.

\begin{itemize}
\tightlist
\item
  \textbf{Python}
\end{itemize}

\begin{Shaded}
\begin{Highlighting}[]
\ImportTok{import}\NormalTok{ numpy }\ImportTok{as}\NormalTok{ np}
\end{Highlighting}
\end{Shaded}

Definimos las matrices de coeficientes, de términos independientes y
ampliada.

\begin{Shaded}
\begin{Highlighting}[]
\NormalTok{A }\OperatorTok{=}\NormalTok{ np.array([[}\OperatorTok{-}\DecValTok{2}\NormalTok{,}\DecValTok{2}\NormalTok{,}\OperatorTok{-}\DecValTok{1}\NormalTok{,}\DecValTok{1}\NormalTok{,}\DecValTok{0}\NormalTok{,}\DecValTok{4}\NormalTok{,}\DecValTok{0}\NormalTok{],[}\OperatorTok{-}\DecValTok{1}\NormalTok{,}\OperatorTok{-}\DecValTok{3}\NormalTok{,}\DecValTok{0}\NormalTok{,}\DecValTok{0}\NormalTok{,}\OperatorTok{-}\DecValTok{1}\NormalTok{,}\DecValTok{5}\NormalTok{,}\OperatorTok{-}\DecValTok{2}\NormalTok{],[}\DecValTok{0}\NormalTok{,}\OperatorTok{-}\DecValTok{1}\NormalTok{,}\DecValTok{3}\NormalTok{,}\OperatorTok{-}\DecValTok{1}\NormalTok{,}\DecValTok{0}\NormalTok{,}\DecValTok{0}\NormalTok{,}\DecValTok{0}\NormalTok{],[}\DecValTok{0}\NormalTok{,}\DecValTok{1}\NormalTok{,}\DecValTok{0}\NormalTok{,}\DecValTok{3}\NormalTok{,}\OperatorTok{-}\DecValTok{2}\NormalTok{,}\DecValTok{1}\NormalTok{,}\DecValTok{4}\NormalTok{],[}\DecValTok{0}\NormalTok{,}\OperatorTok{-}\DecValTok{3}\NormalTok{,}\DecValTok{0}\NormalTok{,}\OperatorTok{-}\DecValTok{5}\NormalTok{,}\DecValTok{3}\NormalTok{,}\DecValTok{0}\NormalTok{,}\OperatorTok{-}\DecValTok{2}\NormalTok{],[}\OperatorTok{-}\DecValTok{3}\NormalTok{,}\DecValTok{1}\NormalTok{,}\DecValTok{1}\NormalTok{,}\DecValTok{1}\NormalTok{,}\OperatorTok{-}\DecValTok{2}\NormalTok{,}\DecValTok{2}\NormalTok{,}\DecValTok{1}\NormalTok{]])}

\NormalTok{b }\OperatorTok{=}\NormalTok{ np.array([}\DecValTok{5}\NormalTok{,}\DecValTok{3}\NormalTok{,}\DecValTok{5}\NormalTok{,}\DecValTok{0}\NormalTok{,}\DecValTok{5}\NormalTok{,}\DecValTok{0}\NormalTok{])}

\NormalTok{AB }\OperatorTok{=}\NormalTok{ np.array([[}\OperatorTok{-}\DecValTok{2}\NormalTok{,}\DecValTok{2}\NormalTok{,}\OperatorTok{-}\DecValTok{1}\NormalTok{,}\DecValTok{1}\NormalTok{,}\DecValTok{0}\NormalTok{,}\DecValTok{4}\NormalTok{,}\DecValTok{0}\NormalTok{,}\DecValTok{5}\NormalTok{],[}\OperatorTok{-}\DecValTok{1}\NormalTok{,}\OperatorTok{-}\DecValTok{3}\NormalTok{,}\DecValTok{0}\NormalTok{,}\DecValTok{0}\NormalTok{,}\OperatorTok{-}\DecValTok{1}\NormalTok{,}\DecValTok{5}\NormalTok{,}\OperatorTok{-}\DecValTok{2}\NormalTok{,}\DecValTok{3}\NormalTok{],[}\DecValTok{0}\NormalTok{,}\OperatorTok{-}\DecValTok{1}\NormalTok{,}\DecValTok{3}\NormalTok{,}\OperatorTok{-}\DecValTok{1}\NormalTok{,}\DecValTok{0}\NormalTok{,}\DecValTok{0}\NormalTok{,}\DecValTok{0}\NormalTok{,}\DecValTok{5}\NormalTok{],[}\DecValTok{0}\NormalTok{,}\DecValTok{1}\NormalTok{,}\DecValTok{0}\NormalTok{,}\DecValTok{3}\NormalTok{,}\OperatorTok{-}\DecValTok{2}\NormalTok{,}\DecValTok{1}\NormalTok{,}\DecValTok{4}\NormalTok{,}\DecValTok{0}\NormalTok{],[}\DecValTok{0}\NormalTok{,}\OperatorTok{-}\DecValTok{3}\NormalTok{,}\DecValTok{0}\NormalTok{,}\OperatorTok{-}\DecValTok{5}\NormalTok{,}\DecValTok{3}\NormalTok{,}\DecValTok{0}\NormalTok{,}\OperatorTok{-}\DecValTok{2}\NormalTok{,}\DecValTok{5}\NormalTok{],[}\OperatorTok{-}\DecValTok{3}\NormalTok{,}\DecValTok{1}\NormalTok{,}\DecValTok{1}\NormalTok{,}\DecValTok{1}\NormalTok{,}\OperatorTok{-}\DecValTok{2}\NormalTok{,}\DecValTok{2}\NormalTok{,}\DecValTok{1}\NormalTok{,}\DecValTok{1}\NormalTok{]])}
\end{Highlighting}
\end{Shaded}

Vemos si el rango de la ampliada es igual que el de la matriz de
coeficientes; y si es igual al nº de incognitas.

\begin{Shaded}
\begin{Highlighting}[]
\NormalTok{np.linalg.matrix_rank(A) }\OperatorTok{==}\NormalTok{ np.linalg.matrix_rank(AB)}
\end{Highlighting}
\end{Shaded}

\begin{verbatim}
## True
\end{verbatim}

\begin{Shaded}
\begin{Highlighting}[]
\NormalTok{np.linalg.matrix_rank(A) }\OperatorTok{==} \DecValTok{7}
\end{Highlighting}
\end{Shaded}

\begin{verbatim}
## False
\end{verbatim}

Vemos que los rangos de las matrices comparadas son iguales, sin embargo
el número de incógnitas es diferente al rango por lo que estamos ante un
sistema compatible indeterminado.

Ahora procedemos a encontrar cual es la solución al sistema.

\begin{Shaded}
\begin{Highlighting}[]
\ImportTok{from}\NormalTok{ sympy }\ImportTok{import} \OperatorTok{*}
\ImportTok{from}\NormalTok{ sympy.solvers.solveset }\ImportTok{import}\NormalTok{ linsolve}

\NormalTok{x1,x2,x3,x4,x5,x6,x7 }\OperatorTok{=}\NormalTok{ symbols(}\StringTok{'x1,x2,x3,x4,x5,x6,x7'}\NormalTok{)}

\NormalTok{linsolve(Matrix(([}\OperatorTok{-}\DecValTok{2}\NormalTok{,}\DecValTok{2}\NormalTok{,}\OperatorTok{-}\DecValTok{1}\NormalTok{,}\DecValTok{1}\NormalTok{,}\DecValTok{0}\NormalTok{,}\DecValTok{4}\NormalTok{,}\DecValTok{0}\NormalTok{,}\DecValTok{5}\NormalTok{],[}\OperatorTok{-}\DecValTok{1}\NormalTok{,}\OperatorTok{-}\DecValTok{3}\NormalTok{,}\DecValTok{0}\NormalTok{,}\DecValTok{0}\NormalTok{,}\OperatorTok{-}\DecValTok{1}\NormalTok{,}\DecValTok{5}\NormalTok{,}\OperatorTok{-}\DecValTok{2}\NormalTok{,}\DecValTok{3}\NormalTok{],[}\DecValTok{0}\NormalTok{,}\OperatorTok{-}\DecValTok{1}\NormalTok{,}\DecValTok{3}\NormalTok{,}\OperatorTok{-}\DecValTok{1}\NormalTok{,}\DecValTok{0}\NormalTok{,}\DecValTok{0}\NormalTok{,}\DecValTok{0}\NormalTok{,}\DecValTok{5}\NormalTok{],[}\DecValTok{0}\NormalTok{,}\DecValTok{1}\NormalTok{,}\DecValTok{0}\NormalTok{,}\DecValTok{3}\NormalTok{,}\OperatorTok{-}\DecValTok{2}\NormalTok{,}\DecValTok{1}\NormalTok{,}\DecValTok{4}\NormalTok{,}\DecValTok{0}\NormalTok{],[}\DecValTok{0}\NormalTok{,}\OperatorTok{-}\DecValTok{3}\NormalTok{,}\DecValTok{0}\NormalTok{,}\OperatorTok{-}\DecValTok{5}\NormalTok{,}\DecValTok{3}\NormalTok{,}\DecValTok{0}\NormalTok{,}\OperatorTok{-}\DecValTok{2}\NormalTok{,}\DecValTok{5}\NormalTok{],[}\OperatorTok{-}\DecValTok{3}\NormalTok{,}\DecValTok{1}\NormalTok{,}\DecValTok{1}\NormalTok{,}\DecValTok{1}\NormalTok{,}\OperatorTok{-}\DecValTok{2}\NormalTok{,}\DecValTok{2}\NormalTok{,}\DecValTok{1}\NormalTok{,}\DecValTok{1}\NormalTok{])),(x1,x2,x3,x4,x5,x6,x7))}
\end{Highlighting}
\end{Shaded}

\begin{verbatim}
## FiniteSet((68 - 621*x7/7, 56 - 509*x7/7, 233*x7/14 - 11, 1717*x7/14 - 94, 1853*x7/14 - 99, 28 - 69*x7/2, x7))
\end{verbatim}

Por tanto, la solución del sistema es S =
(\(68 - \frac{621x_7}{7}\),\(56 - \frac{509x_7}{7}\),
\(-11 + \frac{233x_7}{14}\), \(-94 + \frac{1717x_7}{14}\),
\(-99 + \frac{1853x_7}{14}\), \(28 - \frac{69x_7}{2}\), \(x_7\)), siendo
\(x_7\) la variable indeterminada.

\hypertarget{ejercicio-3}{%
\section{Ejercicio 3}\label{ejercicio-3}}

Resolver el siguiente sistema del pdf.

Primero, comprobar el tipo de sistema (compatible determinado,
compatible indeterminado o incompatible) con R, Python y Octave.

Después, en caso de haber solución, calcularla con R, Python y Octave.
Finalmente, indicar la solución final junto con el procedimiento llevado
a cabo.

\begin{itemize}
\tightlist
\item
  \textbf{R}
\end{itemize}

\begin{Shaded}
\begin{Highlighting}[]
\KeywordTok{library}\NormalTok{(matlib)}
\end{Highlighting}
\end{Shaded}

Definimos las matrices de coeficientes, de términos independientes y
ampliada.

\begin{Shaded}
\begin{Highlighting}[]
\NormalTok{A =}\StringTok{ }\KeywordTok{rbind}\NormalTok{(}\KeywordTok{c}\NormalTok{(}\DecValTok{10}\NormalTok{,}\DecValTok{2}\NormalTok{,}\OperatorTok{-}\DecValTok{1}\NormalTok{,}\DecValTok{1}\NormalTok{,}\DecValTok{0}\NormalTok{,}\DecValTok{10}\NormalTok{), }\KeywordTok{c}\NormalTok{(}\OperatorTok{-}\DecValTok{1}\NormalTok{,}\OperatorTok{-}\DecValTok{3}\NormalTok{,}\DecValTok{0}\NormalTok{,}\DecValTok{0}\NormalTok{,}\OperatorTok{-}\DecValTok{1}\NormalTok{,}\DecValTok{5}\NormalTok{), }\KeywordTok{c}\NormalTok{(}\DecValTok{9}\NormalTok{,}\OperatorTok{-}\DecValTok{1}\NormalTok{,}\OperatorTok{-}\DecValTok{1}\NormalTok{,}\DecValTok{1}\NormalTok{,}\OperatorTok{-}\DecValTok{1}\NormalTok{,}\DecValTok{15}\NormalTok{), }\KeywordTok{c}\NormalTok{(}\DecValTok{17}\NormalTok{,}\DecValTok{1}\NormalTok{,}\DecValTok{0}\NormalTok{,}\DecValTok{3}\NormalTok{,}\DecValTok{5}\NormalTok{,}\OperatorTok{-}\DecValTok{15}\NormalTok{), }\KeywordTok{c}\NormalTok{(}\DecValTok{0}\NormalTok{,}\OperatorTok{-}\DecValTok{10}\NormalTok{,}\DecValTok{0}\NormalTok{,}\OperatorTok{-}\DecValTok{5}\NormalTok{,}\DecValTok{3}\NormalTok{,}\DecValTok{0}\NormalTok{), }\KeywordTok{c}\NormalTok{(}\OperatorTok{-}\DecValTok{3}\NormalTok{,}\DecValTok{1}\NormalTok{,}\DecValTok{1}\NormalTok{,}\DecValTok{1}\NormalTok{,}\OperatorTok{-}\DecValTok{2}\NormalTok{,}\DecValTok{2}\NormalTok{))}

\NormalTok{b =}\StringTok{ }\KeywordTok{c}\NormalTok{(}\DecValTok{0}\NormalTok{,}\DecValTok{5}\NormalTok{,}\DecValTok{0}\NormalTok{,}\DecValTok{4}\NormalTok{,}\OperatorTok{-}\DecValTok{21}\NormalTok{,}\DecValTok{11}\NormalTok{)}

\NormalTok{AB =}\StringTok{ }\KeywordTok{cbind}\NormalTok{(A,b)}
\end{Highlighting}
\end{Shaded}

Vemos si el rango de la ampliada es igual que el de la matriz de
coeficientes; y si es igual al nº de incognitas.

\begin{Shaded}
\begin{Highlighting}[]
\KeywordTok{R}\NormalTok{(A) }\OperatorTok{==}\StringTok{ }\KeywordTok{R}\NormalTok{(AB)}
\end{Highlighting}
\end{Shaded}

\begin{verbatim}
## [1] FALSE
\end{verbatim}

\begin{Shaded}
\begin{Highlighting}[]
\KeywordTok{R}\NormalTok{(A) }\OperatorTok{==}\StringTok{ }\DecValTok{6}
\end{Highlighting}
\end{Shaded}

\begin{verbatim}
## [1] FALSE
\end{verbatim}

Vemos que ambas hipótesis no se cumplen por lo que estamos ante un
sistema incompatible.

No existe solución.

\begin{itemize}
\tightlist
\item
  \textbf{Python}
\end{itemize}

\begin{Shaded}
\begin{Highlighting}[]
\ImportTok{import}\NormalTok{ numpy }\ImportTok{as}\NormalTok{ np}
\end{Highlighting}
\end{Shaded}

Definimos las matrices de coeficientes, de términos independientes y
ampliada.

\begin{Shaded}
\begin{Highlighting}[]
\NormalTok{A }\OperatorTok{=}\NormalTok{ np.array([[}\DecValTok{10}\NormalTok{,}\DecValTok{2}\NormalTok{,}\OperatorTok{-}\DecValTok{1}\NormalTok{,}\DecValTok{1}\NormalTok{,}\DecValTok{0}\NormalTok{,}\DecValTok{10}\NormalTok{],[}\OperatorTok{-}\DecValTok{1}\NormalTok{,}\OperatorTok{-}\DecValTok{3}\NormalTok{,}\DecValTok{0}\NormalTok{,}\DecValTok{0}\NormalTok{,}\OperatorTok{-}\DecValTok{1}\NormalTok{,}\DecValTok{5}\NormalTok{],[}\DecValTok{9}\NormalTok{,}\OperatorTok{-}\DecValTok{1}\NormalTok{,}\OperatorTok{-}\DecValTok{1}\NormalTok{,}\DecValTok{1}\NormalTok{,}\OperatorTok{-}\DecValTok{1}\NormalTok{,}\DecValTok{15}\NormalTok{],[}\DecValTok{17}\NormalTok{,}\DecValTok{1}\NormalTok{,}\DecValTok{0}\NormalTok{,}\DecValTok{3}\NormalTok{,}\DecValTok{5}\NormalTok{,}\OperatorTok{-}\DecValTok{15}\NormalTok{],[}\DecValTok{0}\NormalTok{,}\OperatorTok{-}\DecValTok{10}\NormalTok{,}\DecValTok{0}\NormalTok{,}\OperatorTok{-}\DecValTok{5}\NormalTok{,}\DecValTok{3}\NormalTok{,}\DecValTok{0}\NormalTok{],[}\OperatorTok{-}\DecValTok{3}\NormalTok{,}\DecValTok{1}\NormalTok{,}\DecValTok{1}\NormalTok{,}\DecValTok{1}\NormalTok{,}\OperatorTok{-}\DecValTok{2}\NormalTok{,}\DecValTok{2}\NormalTok{]])}

\NormalTok{b }\OperatorTok{=}\NormalTok{ np.array([}\DecValTok{0}\NormalTok{,}\DecValTok{5}\NormalTok{,}\DecValTok{0}\NormalTok{,}\DecValTok{4}\NormalTok{,}\OperatorTok{-}\DecValTok{21}\NormalTok{,}\DecValTok{11}\NormalTok{])}

\NormalTok{AB }\OperatorTok{=}\NormalTok{ np.array([[}\DecValTok{10}\NormalTok{,}\DecValTok{2}\NormalTok{,}\OperatorTok{-}\DecValTok{1}\NormalTok{,}\DecValTok{1}\NormalTok{,}\DecValTok{0}\NormalTok{,}\DecValTok{10}\NormalTok{,}\DecValTok{0}\NormalTok{],[}\OperatorTok{-}\DecValTok{1}\NormalTok{,}\OperatorTok{-}\DecValTok{3}\NormalTok{,}\DecValTok{0}\NormalTok{,}\DecValTok{0}\NormalTok{,}\OperatorTok{-}\DecValTok{1}\NormalTok{,}\DecValTok{5}\NormalTok{,}\DecValTok{5}\NormalTok{],[}\DecValTok{9}\NormalTok{,}\OperatorTok{-}\DecValTok{1}\NormalTok{,}\OperatorTok{-}\DecValTok{1}\NormalTok{,}\DecValTok{1}\NormalTok{,}\OperatorTok{-}\DecValTok{1}\NormalTok{,}\DecValTok{15}\NormalTok{,}\DecValTok{0}\NormalTok{],[}\DecValTok{17}\NormalTok{,}\DecValTok{1}\NormalTok{,}\DecValTok{0}\NormalTok{,}\DecValTok{3}\NormalTok{,}\DecValTok{5}\NormalTok{,}\OperatorTok{-}\DecValTok{15}\NormalTok{,}\DecValTok{4}\NormalTok{],[}\DecValTok{0}\NormalTok{,}\OperatorTok{-}\DecValTok{10}\NormalTok{,}\DecValTok{0}\NormalTok{,}\OperatorTok{-}\DecValTok{5}\NormalTok{,}\DecValTok{3}\NormalTok{,}\DecValTok{0}\NormalTok{,}\OperatorTok{-}\DecValTok{21}\NormalTok{],[}\OperatorTok{-}\DecValTok{3}\NormalTok{,}\DecValTok{1}\NormalTok{,}\DecValTok{1}\NormalTok{,}\DecValTok{1}\NormalTok{,}\OperatorTok{-}\DecValTok{2}\NormalTok{,}\DecValTok{2}\NormalTok{,}\DecValTok{11}\NormalTok{]])}
\end{Highlighting}
\end{Shaded}

Vemos si el rango de la ampliada es igual que el de la matriz de
coeficientes; y si es igual al nº de incognitas.

\begin{Shaded}
\begin{Highlighting}[]
\NormalTok{np.linalg.matrix_rank(A) }\OperatorTok{==}\NormalTok{ np.linalg.matrix_rank(AB)}
\end{Highlighting}
\end{Shaded}

\begin{verbatim}
## False
\end{verbatim}

\begin{Shaded}
\begin{Highlighting}[]
\NormalTok{np.linalg.matrix_rank(A) }\OperatorTok{==} \DecValTok{6}
\end{Highlighting}
\end{Shaded}

\begin{verbatim}
## False
\end{verbatim}

Vemos que ambas hipótesis no se cumplen por lo que estamos ante un
sistema incompatible.

No existe solución.

\hypertarget{ejercicio-4}{%
\section{Ejercicio 4}\label{ejercicio-4}}

Encuentra la matriz X, ya sea a mano o con R,Python u Octave, tal que:

\begin{itemize}
\tightlist
\item
  APARTADO A
\end{itemize}

\[AX+3B=−5X\]

A y B aparecen en el PDF adjunto. Las definimos:

\begin{Shaded}
\begin{Highlighting}[]
\NormalTok{A =}\StringTok{ }\KeywordTok{rbind}\NormalTok{(}\KeywordTok{c}\NormalTok{(}\OperatorTok{-}\DecValTok{6}\NormalTok{,}\OperatorTok{-}\DecValTok{3}\NormalTok{),}\KeywordTok{c}\NormalTok{(}\DecValTok{0}\NormalTok{,}\OperatorTok{-}\DecValTok{3}\NormalTok{))}
\NormalTok{B =}\StringTok{ }\KeywordTok{rbind}\NormalTok{(}\KeywordTok{c}\NormalTok{(}\OperatorTok{-}\DecValTok{1}\NormalTok{,}\DecValTok{0}\NormalTok{),}\KeywordTok{c}\NormalTok{(}\DecValTok{4}\NormalTok{,}\OperatorTok{-}\DecValTok{2}\NormalTok{))}
\NormalTok{I =}\StringTok{ }\KeywordTok{rbind}\NormalTok{(}\KeywordTok{c}\NormalTok{(}\DecValTok{1}\NormalTok{,}\DecValTok{0}\NormalTok{),}\KeywordTok{c}\NormalTok{(}\DecValTok{0}\NormalTok{,}\DecValTok{1}\NormalTok{))}
\end{Highlighting}
\end{Shaded}

Recolocamos la ecuación del siguiente modo:

\[(A+5I)X = -3B\] Definimos \(M=A+5I\) y \(N=-3B\).

\begin{Shaded}
\begin{Highlighting}[]
\NormalTok{M =}\StringTok{ }\NormalTok{A}\OperatorTok{+}\NormalTok{(}\DecValTok{5}\OperatorTok{*}\NormalTok{I)}
\NormalTok{N =}\StringTok{ }\DecValTok{-3}\OperatorTok{*}\NormalTok{B}
\end{Highlighting}
\end{Shaded}

Determinamos el valor de X.

\begin{Shaded}
\begin{Highlighting}[]
\NormalTok{X =}\StringTok{ }\KeywordTok{solve}\NormalTok{(M,N)}
\NormalTok{X}
\end{Highlighting}
\end{Shaded}

\begin{verbatim}
##      [,1] [,2]
## [1,]   15   -9
## [2,]   -6    3
\end{verbatim}

\begin{Shaded}
\begin{Highlighting}[]
\NormalTok{(A }\OperatorTok\StringTok{ }\NormalTok{X)  }\OperatorTok{+}\StringTok{ }\NormalTok{(}\DecValTok{3}\OperatorTok{*}\NormalTok{B) }\OperatorTok{==}\StringTok{ }\NormalTok{(}\OperatorTok{-}\DecValTok{5}\OperatorTok{*}\NormalTok{X)}
\end{Highlighting}
\end{Shaded}

\begin{verbatim}
##      [,1] [,2]
## [1,] TRUE TRUE
## [2,] TRUE TRUE
\end{verbatim}

\begin{itemize}
\tightlist
\item
  APARTADO B
\end{itemize}

\[ACX+3B=10I−X\] Las matrices A, B, C e I se definen en el PDF adjunto.

\begin{Shaded}
\begin{Highlighting}[]
\NormalTok{A =}\StringTok{ }\KeywordTok{rbind}\NormalTok{(}\KeywordTok{c}\NormalTok{(}\DecValTok{1}\NormalTok{,}\DecValTok{2}\NormalTok{),}\KeywordTok{c}\NormalTok{(}\DecValTok{0}\NormalTok{,}\DecValTok{2}\NormalTok{))}
\NormalTok{B =}\StringTok{ }\KeywordTok{rbind}\NormalTok{(}\KeywordTok{c}\NormalTok{(}\DecValTok{2}\NormalTok{,}\OperatorTok{-}\DecValTok{1}\NormalTok{),}\KeywordTok{c}\NormalTok{(}\OperatorTok{-}\DecValTok{1}\NormalTok{,}\DecValTok{5}\NormalTok{))}
\NormalTok{C =}\StringTok{ }\KeywordTok{rbind}\NormalTok{(}\KeywordTok{c}\NormalTok{(}\DecValTok{2}\NormalTok{,}\DecValTok{6}\NormalTok{),}\KeywordTok{c}\NormalTok{(}\OperatorTok{-}\DecValTok{1}\NormalTok{,}\OperatorTok{-}\FloatTok{0.5}\NormalTok{))}
\NormalTok{I =}\StringTok{ }\KeywordTok{rbind}\NormalTok{(}\KeywordTok{c}\NormalTok{(}\DecValTok{1}\NormalTok{,}\DecValTok{0}\NormalTok{),}\KeywordTok{c}\NormalTok{(}\DecValTok{0}\NormalTok{,}\DecValTok{1}\NormalTok{))}
\end{Highlighting}
\end{Shaded}

Recolocamos la ecuación del siguiente modo:

\[(AC+I)X=10I-3B\]

Definimos \(M=AC+I\) y \(N=10I-3B\).

\begin{Shaded}
\begin{Highlighting}[]
\NormalTok{M =}\StringTok{ }\NormalTok{(A}\OperatorTok\NormalTok{C)}\OperatorTok{+}\NormalTok{I}
\NormalTok{N=}\DecValTok{10}\OperatorTok{*}\NormalTok{I}\DecValTok{-3}\OperatorTok{*}\NormalTok{B}
\end{Highlighting}
\end{Shaded}

Determinamos el valor de X.

\begin{Shaded}
\begin{Highlighting}[]
\NormalTok{X =}\KeywordTok{solve}\NormalTok{(M,N)}
\NormalTok{X}
\end{Highlighting}
\end{Shaded}

\begin{verbatim}
##      [,1] [,2]
## [1,] -1.5  2.5
## [2,]  1.1  0.1
\end{verbatim}

\begin{Shaded}
\begin{Highlighting}[]
\NormalTok{A}\OperatorTok\NormalTok{C}\OperatorTok\NormalTok{X}\OperatorTok{+}\DecValTok{3}\OperatorTok{*}\NormalTok{B }\OperatorTok{==}\StringTok{ }\DecValTok{10}\OperatorTok{*}\NormalTok{I }\OperatorTok{-}\StringTok{ }\NormalTok{X}
\end{Highlighting}
\end{Shaded}

\begin{verbatim}
##      [,1] [,2]
## [1,] TRUE TRUE
## [2,] TRUE TRUE
\end{verbatim}

\end{document}
